% ---------------------------- Problem 1----------------------------------
\subsubsection*{\center Задача № 1.}
{\bf Условие.~}
Разложить в ряд Фурье заданную функцию $f(x)$, построить графики $f(x)$ и суммы ее ряда Фурье. Если не указывается, какой вид разложения в ряд необходимо представить, то требуетчя разложить функцию либо в общий тригонометрический ряд Фурье, либо следует выбрать оптимальный вид разложения в зависимости от данной функции.
\begin{center}

	$f(x)=
	\left\{
	\begin{array}{cc}
	0, 		&	-\pi \leq x<0, \\ 
	x^{2}, 	& 	0 \leq x \leq \pi. 
	\end{array}\right. \ \ $ на отрезке $[-\pi ; \pi]  $ 
\end{center}
{\bf Решение.~}	
%График
\begin{center}
	\begin{minipage}[t][6.75cm][c]{6.75cm}
		\begin{tikzpicture}[
		declare function={
			func(\x)=
			and(\x >= pi, \x <0) * (0.0) + 
			(\x >= 0, \x <=pi) * (\x^2);
		}
		]
		\begin{axis}[
		axis x line=middle, axis y line=middle,
		axis equal,	
		ymin=-2, ymax=11, ytick={-1,pi*pi}, 
		yticklabels={-1,$\pi^2$},
		ylabel=$y$,
		xmin=-6.38, xmax=6.38, xtick={-2*pi,-pi,pi,2*pi},
		xticklabels={$-2\pi$,$-\pi$,$\pi$,$2\pi$},
		xlabel=$x$,
		domain=0.0:pi,samples=600 % added		
		]
		
		\addplot [domain=-pi:0,black,line width=2pt] {0.0};
		\addplot [domain=0:pi,black,line width=2pt] {x^2};	
		\end{axis}
		\end{tikzpicture}
	\end{minipage}		
\end{center}
\noindent
Построим общий тригонометрический ряд Фурье вида
$$
f(x)=\frac{a_0}{2}+\sum_{n=1}^\infty 
	\left(a_n\cos{(n\omega x)}+b_n\sin{(n\omega x)}\right),\quad\text{где}\,\omega=\frac{2\pi}{T},\,T=2\pi.
$$
\noindent
Вычислим коэффициенты
$$
\begin{array}{rcl}
a_0 &=& \displaystyle\frac{1}{\pi}\left.\left(
\int\limits^0_{-\pi} 0\,dx 
+ \int\limits^{\pi}_0 x^2\,dx \right) = 
\frac{1}{3\pi}\left
x^3\right\right|_0^{\pi} \right = \frac{{\pi}^2}{3},				 \\[12pt]
a_n &=& \displaystyle\frac{1}{\pi}\left(
	\int\limits_{-\pi}^0
	0\cos (nx)\,dx + \int\limits_0^{\pi}
	x^2\cos (nx)\,dx \right) ={}									\\[12pt]
	&=& \displaystyle\frac{1}{\pi}\left(
	\left.\frac{x^2}{n}\sin(nx)\right|_0^{\pi}
	\left.-\frac{2}{n} \left(-\frac{x}{n}\cos(nx)\right|_0^{\pi} 
	+\left.\frac{1}{n^2}\sin (nx) \right|_{\pi}^0\right)\right) = 	\\[12pt]
	&=& \displaystyle\frac{2}{n^2}(-1)^n,\\[12pt]
b_n &=& \displaystyle\frac{1}{\pi}\left(
	\int\limits_{-\pi}^0
	0\sin(nx),dx + \int\limits_0^{\pi}
	x^2\sin(nx),dx \right) ={}									\\[12pt]
	&=& \displaystyle\frac{1}{\pi}\left(
	\left.-\frac{x^2}{n}\cos(nx) \right|_0^{\pi}
	+\left.\frac{2}{n}\left(\frac{x}{n}\sin (nx) \right|_{0}^{\pi} 
	+\left.\frac{1}{n^2}\cos (nx)\right|_0^{\pi}\right)\right) =	\\[12pt]
	&=& \displaystyle-\frac{(-1)^n(2-{\pi}^2n^2)-2}{{\pi}n^3}.
\end{array}
$$
Применив теорему Дирихле о поточечной сходимости ряда Фурье, видим, что построенный ряд Фурье сходится 
к периодическому (с периодом $T=2{\pi}$) продолжению исходной функции при всех $x\ne {\pi}n$, и 
$S({\pi}n)={\pi}^2/2$ при $n=\pm1,\pm3, \pm 5\ldots$, где $S(x)$ --- сумма ряда Ферье. 
График функции $S(x)$ имеет следующий вид
\begin{center}
	\begin{tikzpicture}
	\begin{axis}[xmin=-5*pi, xmax=3*pi, ymin=-1.5, ymax=11,
	width=0.8\textwidth,
	height=0.4\textwidth,
	axis x line=middle,
	axis y line=middle, 
	every axis x label/.style={at={(current axis.right of origin)},anchor=west},
	every inner x axis line/.append style={|-latex'},
	every inner y axis line/.append style={|-latex'},
	minor tick num=1,			
	axis equal=true,
	xtick={-5*pi,-4*pi,-3*pi,-2*pi,-pi,-pi,pi,2*pi,3*pi,4*pi},
		xticklabels={$-5\pi$,$-4\pi$,$-3\pi$,$-2\pi$,$-\pi$,$-\pi$,$\pi$,$2\pi$,$3\pi$,$4\pi$},
		xlabel=$x$,
	xlabel=$x$, 
	ytick={-1,pi*pi/2,pi*pi}, 
		yticklabels={-1,$\frac{\pi^2}{2}$,$\pi^2$},
	ylabel=$S(x)$,          
	samples=100,
	clip=true,
	]
	\addplot[color=black, line width=1.5pt,domain=-5*pi:-4*pi] {0};
	\addplot[color=black, line width=1.5pt,domain=-4*pi:-3*pi] {(\x+4*pi)^2};
	\addplot[color=black, line width=1.5pt,domain=-3*pi:-2*pi] {0};
	\addplot[color=black, line width=1.5pt,domain=-2*pi:-pi]{(\x+2*pi)^2};
	\addplot[color=black, line width=1.5pt,domain=-pi:0] {0};
	\addplot[color=black, line width=1.5pt,domain=0:pi]{\x^2};
	\addplot[color=black, line width=1.5pt,domain=pi:2*pi] {0};
	\addplot[color=black, line width=1.5pt,domain=2*pi:3*pi]{(\x-2*pi)^2};
	\addplot[
	mark=*,
	mark options={fill=black, draw=black},
	only marks,
	] coordinates {(-5*pi, pi^2/2) (-3*pi, pi^2/2) (-pi, pi^2/2) (pi, pi^2/2) (3*pi, pi^2/2)};
	\end{axis}
	\end{tikzpicture}
\end{center}
\noindent
\textbf{Ответ:}
\[
\begin{split}
&f(x) \sim \frac{\pi^2}{6}+\sum_{n=1}^\infty\left[ \displaystyle\frac{2}{n^2}(-1)^n\cos(nx)+\displaystyle\frac{(-1)^n(2-{\pi}^2n^2)-2}{{\pi}n^3}\sin(nx)\right]; \\
&S(\pi n)=\frac{\pi^2}{2}, \text{ при } n=\pm1,\pm3, \pm 5\ldots.
\end{split}
\]




% ---------------------------- Problem 2----------------------------------
\subsubsection*{\center Задача № 2.}
{\bf Условие.~}
Для заданной графически функции $y(x)$ построить ряд Фурье в комплексной форме, изобразить график суммы построенного ряда

%График
\begin{center}
\begin{minipage}[t][6.75cm][c]{6.75cm}
		\begin{tikzpicture}[
		declare function={
			func(\x)=
			and(\x >= 0, \x <= 1.5708) * (1-sin(deg(\x))) + 
			(\x >  1.5708) * 1.0;
		}
		]
		\begin{axis}[
		axis x line=middle, axis y line=middle,
		axis equal,	
		ymin=-1.1, ymax=1.1, ytick={-1,...,1}, ylabel=$y$,
		xmin=-1.1, xmax=5, xtick={-1,1,1.5708,2,3.14159},
		xticklabels={-1,1,$\dfrac{\pi}{2}$,2,$\pi$},
		xlabel=$x$,
		domain=0.0:pi,samples=600 % added		
		]
		
		\addplot [domain=0:1.5708,blue,line width=2pt] {1-sin(deg(\x))};
		\addplot [domain=1.5708:pi,blue,line width=2pt] {1.0};	
		\addplot [dashed, black] coordinates {(1.5708,0)(1.5708,1)};
		\node[] at (axis cs: 1.00,1.25) {\small$1-\sin{x}$};			
		\end{axis}
		\end{tikzpicture}
	\end{minipage}	
\end{center}

\noindent
\textbf{Решение.}\\

\noindent
Ряд Фурье в комплексной форме имеет следующий вид
\[
f(x) = \sum_{n=-\infty}^\infty c_n e^{i\omega nx},\quad c_n=\frac{1}{T}\int\limits_a^b f(x) e^{-i\omega nx}dx,\,\omega=\frac{2\pi}{T}.
\]
В нашем примере $ a=0,b=\pi,T=\pi,\omega=2$, 
найдем коэффицинеты $c_n,\,n=0,\pm1,\pm2,\ldots$
где $\omega=2\pi/T,\,T=\pi.$
$$
\begin{array}{rcl}
c_0 &=&\displaystyle\frac{1}{\pi} \int\limits_0^{\pi} f(x)dx=\frac{a_0}{2}=\frac{\pi-1}{2\pi},\\[12pt]
c_n &=&\displaystyle\frac{1}{\pi}\left(
\int\limits_0^\frac{\pi}{2}
(1-\sin{x})e^{-i2nx}dx+ \int\limits_\frac{\pi}{2}^\pi
e^{-i2nx}dx \right) ={}\\[12pt]
&=&\displaystyle-\frac{4n^2}{\pi(4n^2+1)}\left(
\left.\frac{1}{2in}\sin (x) e^{-2in x} \right|_0^{\pi/2} -\left.\frac{1}{4n^2}\cos (x)\lefte^{-2in x}\right|_0^{\pi/2}\right)
-\frac{1}{2{\pi}in}\left(
\left.e^{-2in x} \right|_0^{\pi/2} -\left.e^{-2in x}\right|_{\pi/2}^{\pi}\right)
= \\[12pt]
&=&\displaystyle\frac{2{\pi}in e^{-2{\pi}in}-2{\pi}in}{4{\pi}^2n^2}+\frac{2{\pi}in e^{-{\pi}in}4n^2+2{\pi}in e^{-{\pi}in}}{{\pi}^2(4n^2+1)^2} - \frac{1}{{\pi}(4n^2+1)} 
=\frac{2ni(-1)^n}{\pi(4n^2+1)}-\frac{1}{\pi(4n^2+1)}.
\end{array}
$$
\noindent
Применив теорему Дирихле о поточечной сходимости ряда Фурье, видим, что построенный ряд Фурье сходится 
к периодическому (с периодом $T=\pi$) продолжению исходной функции при всех $x\ne \frac{\pi n}{2}$, и $S(\frac{\pi n}{2})=1/2$ при 
$n= \pm1,\pm3, \pm5 \ldots$, где $S(x)$ --- сумма ряда Фурье. График функции $S(x)$ имеет вид
\begin{center}
	\begin{tikzpicture}
	\begin{axis}[xmin=-2*pi-1.6, xmax=2*pi+1.6, ymin=-1, ymax=2,
	width=0.8\textwidth,
	height=0.4\textwidth,
	axis x line=middle,
	axis y line=middle, 
	every axis x label/.style={at={(current axis.right of origin)},anchor=west},
	every inner x axis line/.append style={|-latex'},
	every inner y axis line/.append style={|-latex'},
	minor tick num=1,			
	axis equal=true,
		xtick={-5*pi/2,-2*pi,-3*pi/2,-pi,-pi/2,pi/2,pi,3*pi/2,2*pi,5*pi/2},
		xticklabels={$-\displaystyle\frac{5\pi}{2}$,$-2\pi$,$-\displaystyle\frac{3\pi}{2}$,$-\pi$,$-\displaystyle\frac{\pi}{2}$,$\displaystyle\frac{\pi}{2}$,$\pi$,$\displaystyle\frac{3\pi}{2}$,$2\pi$,$\displaystyle\frac{5\pi}{2}$},
	xlabel=$x$, 
		ytick={-1,0.5, 2}, 
		yticklabels={-1,0.5,2},
	ylabel=$S(x)$,          
	samples=100,
	clip=true,
	]
	\addplot[color=black, line width=1.5pt,domain=-5*pi/2:-2*pi] {1};
	\addplot[color=black, line width=1.5pt,domain=-2*pi:-3*pi/2] {1-sin(deg(\x+2*pi))};
	\addplot[color=black, line width=1.5pt,domain=-3*pi/2:-pi]{1};
	\addplot[color=black, line width=1.5pt,domain=-pi:-pi/2] {1-sin(deg(\x+pi))};
	\addplot[color=black, line width=1.5pt,domain=-pi/2:0]{1};
	\addplot[color=black, line width=1.5pt,domain=0:pi/2] {1-sin(deg(\x))};
	\addplot[color=black, line width=1.5pt,domain=pi/2:pi]{1};
	\addplot[color=black, line width=1.5pt,domain=pi:3*pi/2] {1-sin(deg(\x-pi))};
	\addplot[color=black, line width=1.5pt,domain=3*pi/2:2*pi]{1};
		\addplot[color=black, line width=1.5pt,domain=2*pi:5*pi/2] {1-sin(deg(\x-2*pi))};
	\addplot[thick,dashed] coordinates {(-pi,0) (-pi,1)};
	\addplot[thick,dashed] coordinates {(-2*pi,0) (-2*pi,1)};
	\addplot[thick,dashed] coordinates {(pi,0) (pi,1)};
	\addplot[thick,dashed] coordinates {(pi,0) (pi,1)};
	\addplot[thick,dashed] coordinates {(2*pi,0) (2*pi,1)};
	\addplot[
	mark=*,
	mark options={fill=black, draw=black},
	only marks,
	] coordinates {(-pi/2, 0.5) (-3*pi/2, 0.5) (pi/2, 0.5) (3*pi/2, 0.5) (-5*pi/2, 0.5)};
	\end{axis}
	\end{tikzpicture}
\end{center}
\noindent
\textbf{Ответ:}
\[
\begin{split}
&f(x)=\sum_{n=-\infty}^\infty\left( -\frac{1}{{\pi}(4n^2+1)}+{i\frac{2n(-1)^n}{\pi (4n^2+1)}}\right) e^{2nix},~ x\ne \frac{\pi n}{2}; \\
&S(\frac{\pi n}{2})=\frac{1}{2},\quad\text{при}~n\in\mathbb{Z}.
\end{split}
\]
% ---------------------------- Problem 3----------------------------------
\subsubsection*{\center Задача № 3.}
{\bf Условие.~}\\
Найти резольвенту для интегрального уравнения Вольтерры со следующим ядром 
$$K(x,t)=x^2t^2e^\frac{t^5-x^5}{5}.$$
\noindent
{\bf Решение.~}\\
\noindent
Из рекурентных соотношений получаем
$$
\begin{array}{rcl}
K_1(x,t)&=&\displaystyle x^2t^2e^\frac{t^5-x^5}{5}, \\[12pt]
K_2(x,t)&=&\displaystyle\int\limits_t^x K(x,s)K_1(s,t)ds =\int\limits_t^x\displaystyle x^2s^2e^\frac{s^5-x^5}{5} \cdot s^2t^2e^\frac{t^5-s^5}{5}ds = x^2t^2e^{\frac{t^5-x^5}{5}}\cdot{{x^5-t^5}\over{5}},\\[12pt]
K_3(x,t)&=&\displaystyle\int\limits_t^x K(x,s)K_2(s,t)ds = \int\limits_t^x \displaystyle x^2s^2e^\frac{s^5-x^5}{5}\cdot s^2t^2e^{\frac{t^5-s^5}{5}}\cdot{{s^5-t^5}\over{5}} ds = x^2t^2e^{\frac{t^5-x^5}{5}}\cdot\frac{(x^5-t^5)^2}{50}.\\[12pt]
K_4(x,t)&=&\displaystyle\int\limits_t^x K(x,s)K_3(s,t)ds = \int\limits_t^x \displaystyle x^2s^2e^\frac{s^5-x^5}{5}\cdot s^2t^2e^{\frac{t^5-s^5}{5}}\cdot{({s^5-t^5})^2\over{50}} ds = x^2t^2e^{\frac{t^5-x^5}{5}}\cdot\frac{(x^5-t^5)^3}{750}.\\[12pt]
K_j(x,t)&=&\displaystyle x^2t^2e^{\frac{t^5-x^5}{5}} \cdot\displaystyle\frac{(x^5-t^5)^{j-1}}{(j-1)!\cdot5^{j-1}}, \ \ 
j\in{N}.
\end{array}
$$
Подставляя это выражение для итерированных ядер, найдем резольвенту
$$ 
R(x,t,\lambda)=\displaystyle x^2t^2e^{\frac{t^5-x^5}{5}}\sum_{j=1}^\infty \lambda^{j-1}\cdot\displaystyle\frac{(x^5-t^5)^{j-1}}{(j-1)!\cdot5^{j-1}}, \ \ \ 
j=1,2,\ldots
$$
